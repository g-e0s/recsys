\documentclass{article}

\usepackage[T1]{fontenc} % Use 8-bit encoding that has 256 glyphs
\usepackage[utf8]{inputenc}
\usepackage[english,russian]{babel} % Language hyphenation and typographical rules
\usepackage{amssymb}
\usepackage{amsmath}
\usepackage{indentfirst}
\linespread{1.05} % Line spacing - Palatino needs more space between lines
\usepackage{microtype} % Slightly tweak font spacing for aesthetics
\usepackage[hmarginratio=4:3, top=20mm, left=30mm, columnsep=20pt]{geometry} % Document margins
\usepackage[hang, small,labelfont=bf,up,textfont=it,up]{caption} % Custom captions under/above floats in tables or figures
\usepackage{booktabs} % Horizontal rules in tables
\usepackage{lettrine} % The lettrine is the first enlarged letter at the beginning of the text
\usepackage{enumitem} % Customized lists
\setlist[itemize]{noitemsep} % Make itemize lists more compact

\usepackage{titlesec} % Allows customization of titles
\titleformat{\subsection}[block]{\large}{\thesubsection.}{1em}{} % Change the look of the section titles
\usepackage{titling} % Customizing the title section

\usepackage{hyperref} % For hyperlinks in the PDF

%----------------------------------------------------------------------------------------
%	TITLE SECTION
%----------------------------------------------------------------------------------------

\setlength{\droptitle}{-4\baselineskip} % Move the title up

\pretitle{\begin{center}\Huge\bfseries} % Article title formatting
\posttitle{\end{center}} % Article title closing formatting
\title{Recommender systems review} % Article title
\author{%
\textsc{Sarapulov G. V.} \\ % Your name
%\textsc{John Smith}\thanks{A thank you or further information} \\[1ex] % Your name
\normalsize Saint-Petersburg State University \\ % Your institution
\normalsize \href{mailto:john@smith.com}{g-eos@yandex.ru} % Your email address
}
\date{\today} % Leave empty to omit a date
\renewcommand{\maketitlehookd}{%
}

%----------------------------------------------------------------------------------------

\begin{document}

% Print the title
\maketitle

%----------------------------------------------------------------------------------------
%	ARTICLE CONTENTS
%----------------------------------------------------------------------------------------

\section{Collaborative Filtering for Implicit Feedback Datasets}
\subsection{Vocabulary}
\begin{enumerate}
\item implicit feedback - неявная обратная связь
\item substantial - существенный
\item tailored - специально приспособленный
\item abundant - богатый, обильный
\item reluctance - нежелание
\item inherently - по существу, от природы
\item holistic - целостный
\item comprise - содержать, включать
\item strive - стараться, прилагать усилия
\item derive - получить, вывести
\item dense - плотный
\item sparse - разреженный
\item threshold - порог
\item magnitude - величина
\item transfer - переводить
\item exploit - использовать, эксплуатировать
\item contaminate - загрязнять, портить
\item adjustment - поправка
\item 
\end{enumerate}

\subsection{Annotation}
Unique properties of implicit feedback datasets are identified. A new factor model based on treating data as an indication of positive and negative preference with varying confidence levels is proposed. Resulting model is proven to be especially tailored for implicit feedback recommenders. A linearly scalable optimization procedure for proposed algorithm is suggested. In addition, a novel way of explaining recommendations given by the system is offered.


%------------------------------------------------



%----------------------------------------------------------------------------------------
%	REFERENCE LIST
%----------------------------------------------------------------------------------------

\begin{thebibliography}{99} % Bibliography - this is intentionally simple in this template

\bibitem[Yifan Hu et all, 2008]{}
Yifan Hu , Yehuda Koren , Chris Volinsky, Collaborative Filtering for Implicit Feedback Datasets, Proceedings of the 2008 Eighth IEEE International Conference on Data Mining, p.263-272, December 15-19, 2008

\bibitem[Pennington et all]{}
Jeffrey Pennington, Richard Socher, and Christopher D. Manning. GloVe: Global Vectors for Word Representation. 2014
\end{thebibliography}

%----------------------------------------------------------------------------------------

\end{document}
