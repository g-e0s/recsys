\documentclass{article}

\usepackage{blindtext} % Package to generate dummy text throughout this template 
\usepackage[T1]{fontenc}
\usepackage[utf8]{inputenc}
\usepackage[english,russian]{babel} % Language hyphenation and typographical rules
\usepackage{amssymb}
%\usepackage[utf8]{inputenc}
%\usepackage[sc]{mathpazo} % Use the Palatino font
\usepackage[T1]{fontenc} % Use 8-bit encoding that has 256 glyphs
\linespread{1.05} % Line spacing - Palatino needs more space between lines
\usepackage{microtype} % Slightly tweak font spacing for aesthetics

\usepackage[hmarginratio=4:3, top=20mm, left=30mm, columnsep=20pt]{geometry} % Document margins
\usepackage[hang, small,labelfont=bf,up,textfont=it,up]{caption} % Custom captions under/above floats in tables or figures
\usepackage{booktabs} % Horizontal rules in tables

\usepackage{lettrine} % The lettrine is the first enlarged letter at the beginning of the text

\usepackage{enumitem} % Customized lists
\setlist[itemize]{noitemsep} % Make itemize lists more compact

\usepackage{abstract} % Allows abstract customization
\renewcommand{\abstractnamefont}{\normalfont\bfseries} % Set the "Abstract" text to bold
\renewcommand{\abstracttextfont}{\normalfont\small\itshape} % Set the abstract itself to small italic text

\usepackage{titlesec} % Allows customization of titles
\renewcommand\thesection{\Roman{section}} % Roman numerals for the sections
\renewcommand\thesubsection{\roman{subsection}} % roman numerals for subsections
\titleformat{\section}[block]{\large\scshape\centering}{\thesection.}{1em}{} % Change the look of the section titles
\titleformat{\subsection}[block]{\large}{\thesubsection.}{1em}{} % Change the look of the section titles

\usepackage{fancyhdr} % Headers and footers
\pagestyle{fancy} % All pages have headers and footers
\fancyhead{} % Blank out the default header
\fancyfoot{} % Blank out the default footer
\fancyhead[C]{Running title $\bullet$ May 2016 $\bullet$ Vol. XXI, No. 1} % Custom header text
\fancyfoot[RO,LE]{\thepage} % Custom footer text

\usepackage{titling} % Customizing the title section

\usepackage{hyperref} % For hyperlinks in the PDF

%----------------------------------------------------------------------------------------
%	TITLE SECTION
%----------------------------------------------------------------------------------------

\setlength{\droptitle}{-4\baselineskip} % Move the title up

\pretitle{\begin{center}\Huge\bfseries} % Article title formatting
\posttitle{\end{center}} % Article title closing formatting
\title{Рекомендательная система для ретейла: сравнение вероятностной и контентной моделей} % Article title
\author{%
\textsc{Сарапулов Г. В.} \\ % Your name
%\textsc{John Smith}\thanks{A thank you or further information} \\[1ex] % Your name
\normalsize Санкт-Петербургский государственный университет \\ % Your institution
\normalsize Математико-механический факультет \\ % Your institution
\normalsize \href{mailto:john@smith.com}{g-eos@yandex.ru} % Your email address
%\and % Uncomment if 2 authors are required, duplicate these 4 lines if more
%\textsc{Jane Smith}\thanks{Corresponding author} \\[1ex] % Second author's name
%\normalsize University of Utah \\ % Second author's institution
%\normalsize \href{mailto:jane@smith.com}{jane@smith.com} % Second author's email address
}
\date{\today} % Leave empty to omit a date
\renewcommand{\maketitlehookd}{%
\begin{abstract}
%\noindent \blindtext % Dummy abstract text - replace \blindtext with your abstract text
\noindent В работе проведено сравнение двух подходов к построению рекомендательной системы для продуктового ретейла: на основе вероятностной модели и на основе контента. В рамках первого подхода построена вероятностная модель для оценки вероятности покупки товарных групп в зависимости от предыдущих покупок. Для реализации второго подхода построены векторные представления для товарных групп из ассортимента торговой сети и покупательских корзин.
\end{abstract}
}

%----------------------------------------------------------------------------------------

\begin{document}

% Print the title
\maketitle

%----------------------------------------------------------------------------------------
%	ARTICLE CONTENTS
%----------------------------------------------------------------------------------------

\section{Введение}

%\lettrine[nindent=0em,lines=3]{L} orem ipsum dolor sit amet, consectetur adipiscing elit.
%\blindtext % Dummy text

Обозначения:

\begin{itemize}
\item $U$ - множество субъектов (users, покупатели)
\item $I$ - множество объектов (items, товары/товарные группы)
\item $R$ - матрица оценок размера $|U| \times |I|$ (например, $R[u, i] = 1$, если покупатель $u$ купил товар $i$)
\item $x_u$ - вектор признаков субъекта $u$ (демографические признаки, агрегационные данные)
\item $x_i$ - вектор признаков объекта $i$ (характеристики товара)
\item $f: U \times I \rightarrow \hat R$ - функция, сопоставляющая каждой паре $(u, i)$ оценку $\hat r_{u,i}$
\item $L(R, \hat R)$ - функция потерь (например, кросс-энтропия или RMSE)
 \end{itemize}
 
Задача: сформировать список рекомендаций для всех объектов $u \in U$ через нахождение функции $f$, которая минимизирует функцию потерь

\begin{equation}
\label{eq:foo}
f^* = argmin_f  L(R, \hat R)
\end{equation}

В качестве рекомендаций для каждого субъекта выбирается k объектов с наибольшими значениями $\hat r_{u,i}$

%------------------------------------------------

\section{Рекомендательная система на основе наивного байесовского классификатора}

Для каждой товарной группы из ассортимента торговой сети строится модель классификации, оценивающая вероятность покупки данной товарной группы в зависимости от предыдущих покупок.
Пусть $x^{u} = \{x^{u}_{1}, ..., x^{u}_{N}\}$ - вектор признаков покупателя $u$, построенный по истории транзакций, где $x^{u}_{i} = 1$, если покупатель $u$ покупал товар $i$, и $x^{u}_{i} = 0$в противном случае.

%------------------------------------------------

\section{Рекомендательная система на основе векторных представлений товарных групп}

Для каждой товарной группы $i \in I$ из ассортимента торговой сети находим векторное представление $w^{i} \in \mathbb{R}^{K}$.

%Text requiring further explanation\footnote{Example footnote}.

%------------------------------------------------

\section{Результаты}

\begin{table}
\caption{Точность рекомендательных систем (precision at k)}
\centering
\begin{tabular}{llr}
\toprule
\multicolumn{2}{r}{Precision at k} \\
\cmidrule(r){1-3}
Model & Precision at 1 & Precision at 3 \\
\midrule
Naive Bayes & $0.40$ & $0.35$ \\
Item2Vec & $0.37$ & $0.32$ \\
\bottomrule
\end{tabular}
\end{table}

\blindtext % Dummy text

\begin{equation}
\label{eq:emc}
e = mc^2
\end{equation}

\blindtext % Dummy text

%------------------------------------------------

\section{Discussion}

\subsection{Subsection One}

A statement requiring citation \cite{Figueredo:2009dg}.
\blindtext % Dummy text

\subsection{Subsection Two}

\blindtext % Dummy text

%----------------------------------------------------------------------------------------
%	REFERENCE LIST
%----------------------------------------------------------------------------------------

\begin{thebibliography}{99} % Bibliography - this is intentionally simple in this template

\bibitem[Figueredo and Wolf, 2009]{Figueredo:2009dg}
Figueredo, A.~J. and Wolf, P. S.~A. (2009).
\newblock Assortative pairing and life history strategy - a cross-cultural
  study.
\newblock {\em Human Nature}, 20:317--330.
 
\end{thebibliography}

%----------------------------------------------------------------------------------------

\end{document}
